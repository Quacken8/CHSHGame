%Compile by xelatex
\documentclass[11pt,t]{beamer}
\usepackage[utf8]{inputenc}
\usepackage[T1]{fontenc}
\usepackage[czech]{babel}
\usepackage{amsmath}
\usepackage{amsfonts}
\usepackage{amssymb}
\usepackage{graphicx}
%\usepackage{dtk-logos}
\usepackage{color}
\usepackage{microtype}



\usefonttheme{professionalfonts} % using non standard fonts for beamer
\usefonttheme{serif} % default family is serif
\usepackage{fontspec}
\usepackage{ebgaramond}



\addtobeamertemplate{navigation symbols}{}{%
    \usebeamerfont{footline}%
    \usebeamercolor[fg]{footline}%
    \hspace{1em}%
    \raisebox{.25em}{\insertframenumber/\inserttotalframenumber}
}

%My own slide environments
\newenvironment{slidecontent}
	{\vspace*{\fill}
	}
	{
	\vspace*{\fill}
}
\newenvironment{slidetitle}
	{\vspace*{0.5cm}\hspace*{.2cm}\Huge
	}
	{
	\vspace*{0.6cm}
}

%An attempt of taming the default behaviour of \item
\useinnertheme{circles}
\setbeamertemplate{itemize item}{\scriptsize\raise1.25pt\hbox{\color{black}{$\bullet$}}}
\setbeamertemplate{itemize subitem}{\tiny\raise1.5pt\hbox{\color{black}{$\bullet$}}}
\setbeamertemplate{itemize subsubitem}{\tiny\raise1.5pt\hbox{\color{black}{$\bullet$}}}

%Margins
\setbeamersize{text margin left=0.2cm,text margin right=0.2cm}

%I can redefine pauser to nothing if I want to get rid of duplicate slides
\newcommand{\pauser}{\pause}

\title{\color{black}%
\Huge
Kvantové posezení\\
úvod
%\large
}
\date{10. 03. 2023}


\graphicspath{{fig/}}


%%%% Macros
\makeatletter
\newcommand*{\shifttext}[2]{%
  \settowidth{\@tempdima}{#2}%
  \makebox[\@tempdima]{\hspace*{#1}#2}%
}
\makeatother

\newcommand{\cheatbox}[3]{%
\shifttext{#1}{\raisebox{#2}[0pt][0pt]{%
#3%
}}%
}


%%%%%% Tips for people who also want to do cool beamer presentations:
%
%1. When creating the presentation, you can only compile one slide at a time:
% https://tbrink.science/blog/2017/05/13/compiling-only-some-slides-of-a-latex-beamer-presentation/
%\includeonlyframes{current}
% To quickly switch between current slides, I use this vim macro
%:nnoremap <f5> ?label=current<cr>F,v3ed/begin{frame<cr>fnfnpzz
%2. All the latex themes are ugly, so I just drew my theme in gimp and include
%it as a background image. Then the slidetitle and slidecontent environment
%serve as space for my content.
%3. How do I make images float? It would be sane to use a similar environment
%to the floating environment in html. But there is no such thing in LaTeX.....
%Solution: the cheatbox environment creates a box with 0 width and 0 height, so that
%it does not disturb the formatting (it feels like cheating, that's why it's called cheatbox).
%raisbox and shifttext is used for positioning.
%4. If you don't know what \only is, look up the overlays section in the beamer manual
% http://tug.ctan.org/macros/latex/contrib/beamer/doc/beameruserguide.pdf
%or search for animation online
%If you know how \only<1> works, then
%+ is like the next number (and increment the number) and . is the same thing, but don't increment.



\begin{document}

{
\usebackgroundtemplate{\includegraphics[width=\paperwidth,height=\paperheight]{Background2.png}}
\begin{frame}[plain]
\begin{slidetitle}
\end{slidetitle}
\begin{slidecontent}
\maketitle
\end{slidecontent}
\end{frame}
}


{
\usebackgroundtemplate{\includegraphics[width=\paperwidth,height=\paperheight]{Background2.png}}
\begin{frame}[plain]
\begin{slidetitle}
\end{slidetitle}
\begin{slidecontent}
\centering
\Huge
\vfill
Představení se navzájem\\
\phantom{.}\\
\Large
Jak se jmenuješ a~co tě zajímá na kvantové mechanice?
\\
\phantom{.}\\
\pause
Teploměr: jak moc toho víš o kvantové mechanice?
\vfill
\end{slidecontent}
\end{frame}
}

{
\usebackgroundtemplate{\includegraphics[width=\paperwidth,height=\paperheight]{Background.png}}
\begin{frame}[plain]
\begin{slidetitle}
Proč organizujeme tuto akci?
\end{slidetitle}
\begin{slidecontent}
\begin{itemize}
\item Nespokojenost s univerzitním systémem
\begin{itemize}
\item Je nastavený jen pro neurotypické lidi.
\item Věda slouží kapitálu a lidé na univerzitě jsou jen nástroje. Chtělx
bychom, aby věda sloužila těm, kteří dělají vědu, a aby bylx v kontaktu s
komunitou, které přímo pomáhají.
\pause
\end{itemize}
\item Věříme, že jiná univerzita je možná.
\begin{itemize}
\item Pluriverzita?!?
\item Inspirace: autonomní škola Zürich, Paulo Freire, anarchismus.
\end{itemize}
\end{itemize}
\end{slidecontent}
\end{frame}
}

{
\usebackgroundtemplate{\includegraphics[width=\paperwidth,height=\paperheight]{Background.png}}
\begin{frame}[plain]
\begin{slidetitle}
Co si slibujeme?
\end{slidetitle}
\begin{slidecontent}
\begin{itemize}
\item Sami se něco naučíme o~QM a~o~jejím učení.
\item Bude to pro všechny zábava.
\item Safe space pro učení se -- na vlastní kůži poznáme, že to \textbf{jde} jinak.
\item Možná se najde někdo, kdo by se chtěl podílet na dalších podobných přednáškách/akcích.
\end{itemize}
\end{slidecontent}
\end{frame}
}

{
\usebackgroundtemplate{\includegraphics[width=\paperwidth,height=\paperheight]{Background.png}}
\begin{frame}[plain]
\begin{slidetitle}
Anarchismus
\end{slidetitle}
\begin{slidecontent}
\begin{itemize}
\item Soulad prostředků a účelu.
\pause
\item Jaké jsou anarchistické prostředky/účely?
\begin{itemize}
\item Bourání hierarchií (i. e. někdo má moc nad někým jiným).
\begin{itemize}
\item Patriarchální nadvláda/sexismus; kredencialismus (někdo má diplom); ale i
větší znalosti či „charisma”
\end{itemize}
\item Vzájemnost (vzájemná pomoc)
\end{itemize}
\end{itemize}
\end{slidecontent}
\end{frame}
}





{
\usebackgroundtemplate{\includegraphics[width=\paperwidth,height=\paperheight]{Background.png}}
\begin{frame}[plain]
\begin{slidetitle}
Plán
\end{slidetitle}
\begin{slidecontent}
\begin{itemize}
\item Tohle představení
\item Blok 1: kvantové stavy
\item Blok 2: CHSH hra
\item Potom: volná diskuse
\item Mezitím: různé příspěvky
\end{itemize}
\end{slidecontent}
\end{frame}
}



\end{document}
