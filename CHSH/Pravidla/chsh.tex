\documentclass[10pt,a4paper]{article}
\usepackage[utf8]{inputenc}
\usepackage[czech]{babel}
\usepackage[T1]{fontenc}
\usepackage{amsmath}
\usepackage{amsfonts}
\usepackage{amssymb}
\usepackage{graphicx}
\usepackage{lmodern}
\usepackage{braket}
\usepackage[top=2cm,bottom=2cm,left=2cm,right=2cm]{geometry}

\begin{document}

Charlie vybere bity $x$ a $y$.
Cíl je uhodnout $x\land y$, což vypadá následovně:
\begin{center}
\begin{tabular}{ c | c || c }
$x$ & $y$ & $x \land y$ \\ 
\hline
 0 & 0 & 0 \\
 0 & 1 & 0 \\
 1 & 0 & 0 \\
 1 & 1 & 1
\end{tabular}
\end{center}

Alice dá $a$ a Bob dá $b$, což složí jejich tip následovně:
\begin{center}
\begin{tabular}{ c | c || c }
$a$ & $b$ & $a \oplus b$ \\ 
\hline
 0 & 0 & 0 \\
 0 & 1 & 1 \\
 1 & 0 & 1 \\
 1 & 1 & 0
\end{tabular}
\end{center}

Tedy, Alice s Bobem vyhrajou, pokud
\begin{align}
a \oplus b = x \land y \,.
\end{align}

\section{Klasické strategie}

Alice a Bob si vyberou $a$ a $b$ tak, aby $a\oplus b$ vždycky bylo $0$.
Potom je šance na výhru $75 \%$.
Mohly by si také vybrat bity tak, aby $a\oplus b$ vždy bylo $1$.
Pak by jejich šance na výhru byla $25 \%$.

Můžou to udělat chytřeji, když Alice využije znalost $x$ a Bob znalost $b$?
Ne. (bez důkazu)

\section{Kvantové strategie}

Alice a Bob nyní používají \textbf{entanglovaný qubit}. Tento qubit má podobu
dvou fotonu -- Alice a Bob mají oba jeden. Prováznost (entanglovanost) těchto
fotonů vysvětlíme později.

U fotonu je důležitá polarizace, můžeme ji vyjádřit následovně:
\begin{align}
P =
\begin{pmatrix}
\alpha \\
\beta
\end{pmatrix}
\,,
\end{align}
%
kde $\alpha^2 + \beta^2 = 1$. Alice ani Bob netuší, jakou má jejich foton polarizaci.

%\subsection{Kvantové stavy}
%
%Kvantový stav značíme takto (ket):
%\begin{align}
%\ket{\psi}
%\end{align}
%%
%nebo takto (bra):
%\begin{align}
%\bra{\psi} \,.
%\end{align}
%
%Kvantový stav je matematický objekt, ve kterém je zapsaná všechna informace o daném systému.
%
%Příklad je qubit. Qubit může mít stav $\ket{0}$ nebo $\ket{1}$ nebo jiný stav: jakoukoliv lineární kombinaci
%\begin{align}
%\ket{\psi} = \alpha \ket{0} + \sqrt{1-\alpha^2} \ket{1}
%\end{align}
%%
%pro $\alpha \in [0,1]$.
%
%Takže třeba:
%\begin{align}
%\ket{\psi} = \frac{1}{\sqrt{2}} (\ket{0} + \ket{1})
%\end{align}
%%
%je stav, kterému se říká \emph{maximálně promíchaný qubit}.
%
%\subsubsection{Skalární součin}
%
%Překryv dvou stavů (jak jednoduše si je lze splést) se počítá pomocí skalárního součinu.
%My ho zde definujeme takto (ignorujeme teď komplexní amplitudy):
%\begin{align}
%\Braket{0|1} = \Braket{1|0} &= 0 \,,\\
%\Braket{1|1} = \Braket{0|0} &= 1 \,,\\
%(\bra{\phi} + \bra{\psi})\ket{\gamma} &= \braket{\phi|\gamma} + \braket{\psi|\gamma} \,.
%\end{align}
%%
%Poslední vlastnosti se říká linearita.
%
%Pravděpodobnost, že si při měření dva stavy spleteme, se vypočítá jako překryv na druhou.
%\begin{align}
%P(\ket{\psi}, \ket{\phi}) = |\braket{\psi|\phi}|^2 \,.
%\end{align}
%
%Jaká je pravděpodobnost, že si spleteme maximálně promíchaný qubit a stav $\ket{0}$?
%
%Jaká je pravděpodobnost, že si spleteme $\ket{0}$ a $\ket{0}$?
%
%\subsection{2 qubity}
%
%Trochu se to zkomplikuje, když představíme 2 qubity. Pak existují tyto 4 stavy:
%\begin{align}
%\ket{0}\otimes\ket{0}\,, \ket{0}\otimes\ket{1} \,, \ket{1}\otimes\ket{0} \,, \ket{1}\otimes\ket{1}\,.
%\end{align}
%%
%Význam těchto stavů je: qubit A ve stavu 0 + qubit B ve stavu 0 atp.
%Překryv těchto stavů je: 1 pro stav sám se sebou, 0 pro různé stavy (nejde si je splést).
%Pak existujou lineární kombinace těchto stavů jako výše (a lin. komb. těchto lin. komb.).
%
%Překryv dvou stavů lze spočítat následovně následovně:
%\begin{align}
%(\bra{\psi}\otimes \ket{\phi} ) (\ket{a}\otimes \ket{b}) = 
%\braket{\psi | a} \times \braket{\phi|b} \,,
%\end{align}
%%
%kde $\times$ je prosté násobení.
%
%\subsection{2D měření}
%
%Alice a Bob sdílí Bellův stav: %TODO intuice
%\begin{align}
%\ket{\Phi} =
%\frac{1}{\sqrt{2}}
%\left( \ket{00} + \ket{11} \right) \,.
%\end{align}
%
%Dejme tomu, že Alice měří, jestli je qubit A ve stavu $1$ a Bob jestli je jeho qubit ve stavu 1.
%Pak překryv bude:
%\begin{align}
%&\left( \bra{1}\otimes\bra{1}\right)
%\frac{1}{\sqrt{2}}
%\left( \ket{0}\otimes\ket{0} + \ket{1}\otimes\ket{1} \right)
%\\&=
%\frac{1}{\sqrt{2}}
%\left( \bra{1}\otimes\bra{1}
% \ket{0}\otimes\ket{0} +  \bra{1}\otimes\bra{1}\ket{1}\otimes\ket{1} \right) \,.
%\\&=
%\frac{1}{\sqrt{2}} \left( 0 + 1\right) \,.
%\end{align}
%%
%Překryv je $1/\sqrt{2}$, takže pravděpodobnost, že naměří stavy $1$ a $1$ je přesně $1/2$.
%
%
%\subsection{Strategie}
%
%Alice a Bob opět sdílí Bellův stav.
%Zde je jejich strategie:
%
%\subsubsection{$x=0$}
%
%Alice měří, jestli její stav je $\bra{0}$. Pokud ano, dá $a=0$, pokud ne, dá $a=1$
%
%\subsubsection{$x=1$}
%
%Alice měří, jestli její stav je $(\bra{0}+\ket{1})/\sqrt{2}$. Pokud ano, dá
%$a=0$, pokud ne, dá $a=1$
%
%\subsubsection{$y=0$}
%
%Bob měří, jestli jeho stav je
%\begin{align}
%\cos (\pi/8 ) \ket {0} + \sin (\pi/8) \ket{1} \,.
%\end{align}
%Dá $b=0$ pokud ano a $b=1$ pokud ne.
%
%\subsubsection{$y=1$}
%
%Bob měří, jestli jeho stav je
%\begin{align}
%\cos (\pi/8 ) \ket {0} - \sin (\pi/8) \ket{1} \,.
%\end{align}
%Dá $b=0$ pokud ano a $b=1$ pokud ne.
%
%
%%\subsection{Šance na úspěch}
%%
%%Dejme tomu, že $x=0$ a $y=0$. Pak musí $a=b$.


\end{document}
